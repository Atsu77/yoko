%
% B4/M2中間発表予稿テンプレ
%
% File : 	 yoko.tex
% Author : 	 山根 惇秀
% Created on 2022/05/12
%
\documentclass[a4j]{jarticle}
%
% パッケージ
%
\usepackage[dvipdfmx]{graphicx}	% 画像表示用
\usepackage{amsmath}				
\usepackage{amsfonts}					% 数学用の特殊フォント用
\usepackage{multicol}					% 二段組
\usepackage{indentfirst}				% 文の最初にインデント
\usepackage{multirow}
\usepackage{hhline}
%
% 印刷領域等の設定
%
\addtolength{\topmargin}{-25mm}
\setlength{\oddsidemargin}{-12mm}
\setlength{\evensidemargin}{-12mm}
\setlength{\textwidth}{185.0mm}   
\setlength{\textheight}{270.0mm}  
\headheight 7mm
\headsep 2mm
\setlength{\columnsep}{8mm}
%
% 行間の倍率を指定(完成してからページ全体に文章が入るよう調整するとき便利:初期値=1.0)
%
\renewcommand{\baselinestretch}{1.0}
%
% ページ番号(発表順序)
%
\pagestyle{plain}
\setcounter{page}{40}				% ここに発表順序を書く
%
% sectionコマンドの書き換え(余分な行間を消して,サイズ変更,不要な場合はコメントアウト)
%
\makeatletter
\def\section{\@startsection {section}{1}{\z@}{-0.8ex plus -1ex minus -.2ex}{0.8 ex plus .2ex}{\large\bf}}
\makeatother 
%
% 本文
%
\begin{document}
\twocolumn[
	% \hfill{xx-xx} \\ %後で発表順を連絡します
	\begin{center}
		{\LARGE \bf タイトル }\\
		~\\
		\vspace{-3mm}
		{\Large 氏名}
	\end{center}
	\vspace{3mm}
]
%
% 文章の全体の文字フォントサイズと,1行を占める文字フォントサイズ
%
\fontsize{9.5pt}{10.5pt}\selectfont
%
% 1章
%
%
%

\section{はじめに}
\section{交通流モデル}

\newpage



%\begin{figure}[htbp]%[tbp]
% 		\begin{center}
%  		\vspace{-3mm}	
%  		\includegraphics[angle=270,scale=0.5]{./figure/Q.eps}
%  		\vspace{-1mm}	
%		\caption{$\eta:1 \rightarrow 10000のシミュレーション結果$}
%		\label{Q}	
%		\vspace{-7mm}	
% 		\end{center}
%\end{figure}
%
%\begin{figure}[htbp]%[tbp]
% 		\begin{center}
%  		\vspace{-3mm}	
%  		\includegraphics[angle=270,scale=0.5]{./figure/R.eps}
%  		\vspace{-1mm}	
%		\caption{$\eta:0.0001 \rightarrow 10000 のシミュレーション結果$}
%		\label{R}	
%		\vspace{-7mm}	
% 		\end{center}
%\end{figure}


\section{特徴量選択}

%
%
%
\section{障害物検知手法}

\newpage
%
% 参考文献
%
\begin{thebibliography}{9}
% 参考文献を書く(学会毎に定義されている書き方に従って書いたほうが良い)
% 例.
% 電子情報通信学会: https://www.ieice.org/jpn/shiori/iss_2.html
% 情報処理学会: http://www.ipsj.or.jp/journal/submit/ronbun_j_prms.html
% 以下は電子情報通信学会の参考文献に従って書いた例

%以下のコマンドを使用して,参考文献を記載
%\bibitem{}

\end{thebibliography}
\end{document}
"